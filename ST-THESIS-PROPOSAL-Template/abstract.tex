%%%%%%%%%%%%%%%%%%%%%%%%%%%%%%%%%%%%%%%%%%%%%%%%%%%%%%%%%%%%%%%%%%%%%%%%%%%%%%%%%%%%%%%%%%%%%%%%%%%%%%
%
%   Filename    : abstract.tex 
%
%   Description : This file will contain your abstract.
%                 
%%%%%%%%%%%%%%%%%%%%%%%%%%%%%%%%%%%%%%%%%%%%%%%%%%%%%%%%%%%%%%%%%%%%%%%%%%%%%%%%%%%%%%%%%%%%%%%%%%%%%%

\begin{abstract}
GetBetter, a telemedicine system, utilizes medical forms as means to collect data from patients. This paper discusses a proposed implementation of a suitable segmentation algorithm that will be used to extract fields from a digital photograph of a medical form. Several segmentation methods will be compared to arrive at the best possible implementation. If necessary the medical form is to be modified to enhance the segmentation. The segmentation process includes an implementation of a database system to store the data, an Optical character recognition module to translate the digital images to string and an android tablet application to take the photos of medical forms and communicate with the server. 

%
%  Do not put citations or quotes in the abract.
%


\begin{flushleft}
\begin{tabular}{lp{4.25in}}
\hspace{-0.5em}\textbf{Keywords:}\hspace{0.25em} & telemedecine, segmentation, optical character recognition \\
\end{tabular}
\end{flushleft}
\end{abstract}
